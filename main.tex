\documentclass{article}
\renewcommand{\contentsname}{Sommaire}
\usepackage[utf8]{inputenc}

\title{Wumpus IA}
\author{Ulysee Brehon, Luis Enrique González Hilario}
\date{May 2020}

\begin{document}

\maketitle
\tableofcontents

\section{Introduction}
La première phase du projet vise à cartographier l'environnement de jeu Wumpus en appliquant les éléments de la logique propositionnelle et les règles du jeu. Ceci afin de révéler exhaustivement la grille.
Notre objectif ? Un seul, dépensez le moins d'or intelligemment (logiquement, bien sûr)

\section{Environnement}

Notre programme est entièrement dans le fichier: \textbf{cartographie.py}
\newline \newline
La variable \textbf{gophersat\_exec} contient le chemin gophersat pour linux. En cas d'exécution sur Windows, remplacez-le simplement par: \textbf{./lib/gophersat}
\newline \newline
Pour l'executer, simplement run \textbf{cartographie.py}

\section{Caractéristiques du programme}

\begin{enumerate}
\item \textbf{Vocabulaire}
Il y a 5 fonctions pour créer des listes de tous les symboles (la quantité dépend du taille de la grille): generate\_wumpus\_voca (W), generate\_stench\_voca (S), generate\_gold\_voca (G), generate\_brise\_voca (B) et generate\_trou\_voca (T)

Le vocabulaire choisi:

\begin{enumerate}
\item \textbf{$W_{i,j}:$} Un wumpus se trouve en (i,j) \newline  
\item \textbf{$B_{i,j}:$} Une brise se trouve en (i,j) \newline  
\item \textbf{$T_{i,j}:$} Un trou se trouve en (i,j) \newline  
\item \textbf{$S_{i,j}:$} Une odeur se trouve en (i,j) \newline
\item \textbf{$G_{i,j}:$} Il y a de l'Or en (i,j) \newline  

\end{enumerate}

\newline  \newline
\item \textbf{Règles}
Nous avons 7 règles pour créer notre "cerveau déductif" dans le jeu de Wumpus: 

\begin{enumerate}
\item \textbf{insert\_only\_one\_wumpus\_regle:} $W_{i=a,j=b} \Rightarrow \land (\neg W_{i\ne a,j\ne b}), \forall (i, j) \in N^{2}$ \newline

\item \textbf{insert\_safety\_regle:} $\neg W_{0,0} \land  \neg T_{0,0}$ \newline

\item \textbf{insert\_trou\_regle:} $(\neg T_{i,j} \lor B_{i-1,j}) \land (\neg T_{i,j} \lor B_{i+1,j}) \land (\neg T_{i,j} \lor B_{i,j-1}) \land (\neg T_{i,j} \lor B_{i.j+1})$ \newline

\item \textbf{insert\_brise\_regle:} $ \neg B_{i,j} \lor T_{i-1,j} \lor T{i+1,j} \lor T_{i,j-1} \lor T_{i,j+1}$ \newline

\item \textbf{insert\_wumpus\_stench\_regle:} $(\neg W_{i,j} \lor S_{i-1,j}) \land (\neg W_{i,j} \lor S_{i+1,j}) \land (\neg W_{i,j} \lor S_{i,j-1}) \land (\neg W_{i,j} \lor S_{i.j+1})$ \newline

\item \textbf{insert\_stench\_regle:} $\neg S_{i,j} \lor W_{i-1,j} \lor W_{i+1,j} \lor W{i,j-1} \lor W{i,j+1}$  \newline

\item \textbf{insert\_une\_menace\_par\_case\_regle:} $(\neg W_{i,j} \lor \neg T_{i,j}) \land (\neg T_{i,j} \lor \neg W_{i,j})$

\end{enumerate}
\end{enumerate}

\section{Fonctions pour aider à la prise de décision}
\begin{itemize}

    \item \textbf{is\_wumpus\_possible}: Tester la satisfiabilité si on ajoute un Wumpus dans la position donnée sous la forme (i,j)
    \item \textbf{is\_trou\_possible}: Tester la satisfiabilité si on ajoute un trou dans la position donnée sous la forme (i,j)
    \item \textbf{get\_implicit\_negative\_facts}: $B_{i,j} \Leftrightarrow (\neg S_{i,j} \land \neg W_{i,j} \land \neg T_{i,j} \land \neg G_{i,j})$

\end{itemize}

\section{Résultats importants}
\begin{itemize}
    \item{\textbf{Satisfiabilité:}} Résultat de gs.solve(), le modèle trouvé entraine ou n'entraine pas de contradiction.
    \item{\textbf{Modèle:}} Modèle retourné par le solveur SAT
    \item{\textbf{Coût en Or:}} Nôtre chiffre sur la grille par défaut (4*4): 366 
\end{document}
